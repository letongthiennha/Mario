\clearpage
\section{User's Manual}
\label{sec:manual}

This section provides installation, setup, and usage instructions for the Super Mario OOP Project.

\subsection{Installation Requirements}
\begin{itemize}
    \item Windows or Linux OS
    \item C++17 compatible compiler (Visual Studio or g++)
    \item raylib graphics library (\url{https://www.raylib.com/})
    \item (Optional) vcpkg for Windows or Makefile/CMake for Linux builds
\end{itemize}

For a detailed guide on setting up raylib with vcpkg on Windows, see:  
\url{https://www.youtube.com/watch?v=UiZGTIYld1M}

\subsection{Building the Application}
\begin{itemize}
    \item \textbf{Windows:} Clone the repository, open the folder in Visual Studio, install raylib via vcpkg, then build and run the project.
    \item \textbf{Linux:} Install raylib (\texttt{sudo apt install libraylib-dev}), then run \texttt{make run} or use the provided CMake files.
\end{itemize}

\subsection{Usage Instructions}
\begin{enumerate}
    \item Launch the application. The main menu will appear.
    \item Use the arrow keys or mouse to navigate the menu.
    \item Select "Start Game" to begin playing.
    \item Choose your character (Mario or Luigi) if prompted.
    \item Use the following controls during gameplay:
    \begin{itemize}
        \item \textbf{Arrow keys:} Move left/right, crouch, or climb.
        \item \textbf{Spacebar:} Jump.
        \item \textbf{Z:} Run or shoot fireball (if powered up).
        \item \textbf{Esc:} Pause the game and access the pause menu.
    \end{itemize}
    \item Collect coins, power-ups, and defeat enemies to progress through the level.
    \item Reach the end of the level to complete it and advance.
    \item Use the pause menu to adjust settings, restart, or return to the main menu.
\end{enumerate}

\vspace{1em}

\subsection{Customization}
\begin{itemize}
    \item Music and sound effect volumes can be adjusted in the settings menu.
    \item Key bindings and color themes are not yet customizable in this version.
\end{itemize}

\subsection{File Structure}
\begin{itemize}
    \item \texttt{resources/Map/} contains level map files in JSON format.
    \item \texttt{resources/Entity/} contains sprites for characters, enemies, and items.
    \item \texttt{resources/SFX/} and \texttt{resources/Music/} contain sound effects and background music.
\end{itemize}

\subsection{Level Format Example}
A level file (e.g., \texttt{map1.json}) contains tile and object layers. Example snippet:
\begin{verbatim}
{
  "layers": [
    {
      "type": "tilelayer",
      "data": [ ... ]
    },
    {
      "type": "objectgroup",
      "name": "Monsters",
      "objects": [
        { "type": "Goomba", "x": 500, "y": 700 },
        { "type": "BanzaiBill", "x": 1200, "y": 650 }
      ]
    }
  ]
}
\end{verbatim}

\subsection{Troubleshooting}
\begin{itemize}
    \item If the game does not launch, ensure raylib is installed and your compiler supports C++17.
    \item For missing textures or sounds, verify the \texttt{resources/} folder is present and complete.
    \item If controls do not respond, check your keyboard layout and focus on the game window.
\end{itemize}