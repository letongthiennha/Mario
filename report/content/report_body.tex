
\clearpage
\begin{flushleft}
\section*{Abstract}
\addcontentsline{toc}{section}{Abstract}
This project presents a desktop platformer game inspired by Super Mario, written in C++ using the raylib graphics library. The game features classic Mario mechanics such as running, jumping, collecting coins, defeating monsters, and progressing through levels. The codebase is modular and object-oriented, supporting extensibility for new levels, enemies, and features.

\section{Introduction}
Platformer games like Mario are iconic in computer science education and game development. This project aims to recreate the Mario experience, providing a hands-on opportunity to learn about game loops, collision detection, resource management, and real-time rendering in C++. The project demonstrates how classic gameplay can be implemented with modern programming practices.

\section{Work Division and Implementation Notes}
\begin{tabular}{ll}
\toprule
Class / Work & Assigned To \\
\midrule
Enemies (Koopa, Goomba, Koopa (Wing), Banzai Bill, Piranha Plant) & Lộc \\
\midrule
Blocks (Item Block, Stone, etc.) & Hữu \\
\midrule
Items (Fire Flower, Coin, Super Mushroom, 1Up Mushroom) & Khải \\
\midrule
Character (Mario, Luigi); World Builder; Collision Handling; Menu Control, Sound, Resource & Nhân \\
\bottomrule
\end{tabular}



\textbf{Important Implementation Notes:}
\begin{itemize}
    \item \textbf{About Inheritance:}
    \begin{itemize}
        \item Every object class must inherit from \texttt{Entity}.
    \end{itemize}
    \item \textbf{About Implement:}
    \begin{itemize}
        \item Function \texttt{Draw} is called separately from function \texttt{updateStateAndPhysics}.
        \item \texttt{updateStateAndPhysics} happens in \textbf{FIXED\_TIME\_STEP} (see Mario class for logic).
    \end{itemize}
    \item \textbf{About Resource:}
    \begin{itemize}
        \item Every texture and sound must load through \texttt{ResourceManager} (Singleton Pattern).
        \item Every sound must play through \texttt{SoundController} (Singleton Pattern).
    \end{itemize}
    \item \textbf{About File Structure:}
    \begin{itemize}
        \item Header files in \texttt{include/}
        \item Source (\texttt{.cpp}) files in \texttt{src/}
        \item Resource files in \texttt{resources/}
    \end{itemize}
\end{itemize}

\begin{flushleft}
\section{Data Storage}
Game data such as levels, tiles, and enemy placements are stored in JSON files and loaded at runtime. In-memory, the game uses C++ classes and STL containers (vectors, maps) to manage entities, resources, and game state. ResourceManager handles textures, sounds, and music, ensuring efficient loading and unloading.

\section{Project Architecture}
\begin{itemize}
    \item \textbf{Language:} C++
    \item \textbf{Graphics:} raylib (cross-platform game graphics library)
    \item \textbf{Key Modules:}
    \begin{itemize}
        \item \textcolor{blue!70!black}{\texttt{main.cpp}} – Main game loop, window and audio management
        \item \textcolor{teal!80!black}{\texttt{StateManager/States}} – Menu, gameplay, and settings states
        \item \textcolor{orange!80!black}{\texttt{Level/Map}} – Level logic, map loading, entity placement
        \item \textcolor{violet!80!black}{\texttt{PlayableCharacter/Monster/Item/Block}} – Entity classes
        \item \textcolor{red!70!black}{\texttt{ResourceManager}} – Texture, sound, and music management
        \item \textcolor{green!60!black}{\texttt{CollisionMediator}} – Handles all collision logic
    \end{itemize}
    \item \textbf{Development:} Modular C++ project, built using CMake or Visual Studio
\end{itemize}

\section{Implementation Details}
\begin{itemize}
    \item \textbf{Player:} Supports running, jumping, power-ups, and firing fireballs.
    \item \textbf{Enemies:} Includes Goomba, BanzaiBill, and others, each with unique movement and collision logic.
    \item \textbf{Items:} Coins, mushrooms, and power-ups with collection and effect logic.
    \item \textbf{Blocks:} Question blocks, breakable blocks, and static tiles.
    \item \textbf{Level System:} Loads map and object data from \textcolor{orange!80!black}{\texttt{JSON}}, spawns entities, and manages sections for performance.
    \item \textbf{Collision:} All entity interactions (player-enemy, player-item, fireball-enemy, etc.) are handled by \textcolor{green!60!black}{\texttt{CollisionMediator}}.
    \item \textbf{Resource Management:} Centralized loading/unloading of textures, sounds, and music.
    \item \textbf{Menus and UI:} Main menu, settings, and pause implemented as separate states.
\end{itemize}

\section{Technical Problems and Solutions}
\begin{itemize}
    \item \textbf{Consistent Physics:} Implemented a fixed time-step update loop to ensure smooth and predictable movement and collision, regardless of frame rate.
    \item \textbf{Resource Loading:} Designed \textcolor{red!70!black}{\texttt{ResourceManager}} to prevent redundant loading and ensure all assets are available when needed.
    \item \textbf{Collision Complexity:} Developed \textcolor{green!60!black}{\texttt{CollisionMediator}} to handle multiple entity types and resolve interactions efficiently.
    \item \textbf{Level Performance:} Used spatial partitioning (sections) to update and check collisions only for nearby entities.
\end{itemize}

\section{Features Demonstration}
\begin{itemize}
    \item Classic Mario gameplay: running, jumping, collecting coins, defeating enemies.
    \item Multiple levels loaded from \textcolor{orange!80!black}{\texttt{JSON}} map files.
    \item Power-ups and score system.
    \item Sound effects and background music.
    \item Pause, settings, and credits menus.
    \item Modular codebase for easy extension (new levels, enemies, items).
\end{itemize}
\end{flushleft}
\end{flushleft}